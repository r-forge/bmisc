\documentclass[letterpaper, 12pt]{article}
\usepackage[nogin]{Sweave}
\newenvironment{Hinput}%
{}%
{}%
\newenvironment{Houtput}%
{}%
{}%
\newenvironment{Hchunk}%
{\vspace{0.5em}\par\begin{flushleft}}%
{\end{flushleft}}%

\usepackage{color}%
 
\newsavebox{\hlnormalsizeboxclosebrace}%
\newsavebox{\hlnormalsizeboxopenbrace}%
\newsavebox{\hlnormalsizeboxbackslash}%
\newsavebox{\hlnormalsizeboxlessthan}%
\newsavebox{\hlnormalsizeboxgreaterthan}%
\newsavebox{\hlnormalsizeboxdollar}%
\newsavebox{\hlnormalsizeboxunderscore}%
\newsavebox{\hlnormalsizeboxand}%
\newsavebox{\hlnormalsizeboxhash}%
\newsavebox{\hlnormalsizeboxat}%
\newsavebox{\hlnormalsizeboxpercent}% 
\newsavebox{\hlnormalsizeboxhat}%
\newsavebox{\hlnormalsizeboxsinglequote}%
\newsavebox{\hlnormalsizeboxbacktick}%

\setbox\hlnormalsizeboxopenbrace=\hbox{\begin{normalsize}\verb.{.\end{normalsize}}%
\setbox\hlnormalsizeboxclosebrace=\hbox{\begin{normalsize}\verb.}.\end{normalsize}}%
\setbox\hlnormalsizeboxlessthan=\hbox{\begin{normalsize}\verb.<.\end{normalsize}}%
\setbox\hlnormalsizeboxdollar=\hbox{\begin{normalsize}\verb.$.\end{normalsize}}%
\setbox\hlnormalsizeboxunderscore=\hbox{\begin{normalsize}\verb._.\end{normalsize}}%
\setbox\hlnormalsizeboxand=\hbox{\begin{normalsize}\verb.&.\end{normalsize}}%
\setbox\hlnormalsizeboxhash=\hbox{\begin{normalsize}\verb.#.\end{normalsize}}%
\setbox\hlnormalsizeboxat=\hbox{\begin{normalsize}\verb.@.\end{normalsize}}%
\setbox\hlnormalsizeboxbackslash=\hbox{\begin{normalsize}\verb.\.\end{normalsize}}%
\setbox\hlnormalsizeboxgreaterthan=\hbox{\begin{normalsize}\verb.>.\end{normalsize}}%
\setbox\hlnormalsizeboxpercent=\hbox{\begin{normalsize}\verb.%.\end{normalsize}}%
\setbox\hlnormalsizeboxhat=\hbox{\begin{normalsize}\verb.^.\end{normalsize}}%
\setbox\hlnormalsizeboxsinglequote=\hbox{\begin{normalsize}\verb.'.\end{normalsize}}%
\setbox\hlnormalsizeboxbacktick=\hbox{\begin{normalsize}\verb.`.\end{normalsize}}%
\setbox\hlnormalsizeboxhat=\hbox{\begin{normalsize}\verb.^.\end{normalsize}}%



\newsavebox{\hltinyboxclosebrace}%
\newsavebox{\hltinyboxopenbrace}%
\newsavebox{\hltinyboxbackslash}%
\newsavebox{\hltinyboxlessthan}%
\newsavebox{\hltinyboxgreaterthan}%
\newsavebox{\hltinyboxdollar}%
\newsavebox{\hltinyboxunderscore}%
\newsavebox{\hltinyboxand}%
\newsavebox{\hltinyboxhash}%
\newsavebox{\hltinyboxat}%
\newsavebox{\hltinyboxpercent}% 
\newsavebox{\hltinyboxhat}%
\newsavebox{\hltinyboxsinglequote}%
\newsavebox{\hltinyboxbacktick}%

\setbox\hltinyboxopenbrace=\hbox{\begin{tiny}\verb.{.\end{tiny}}%
\setbox\hltinyboxclosebrace=\hbox{\begin{tiny}\verb.}.\end{tiny}}%
\setbox\hltinyboxlessthan=\hbox{\begin{tiny}\verb.<.\end{tiny}}%
\setbox\hltinyboxdollar=\hbox{\begin{tiny}\verb.$.\end{tiny}}%
\setbox\hltinyboxunderscore=\hbox{\begin{tiny}\verb._.\end{tiny}}%
\setbox\hltinyboxand=\hbox{\begin{tiny}\verb.&.\end{tiny}}%
\setbox\hltinyboxhash=\hbox{\begin{tiny}\verb.#.\end{tiny}}%
\setbox\hltinyboxat=\hbox{\begin{tiny}\verb.@.\end{tiny}}%
\setbox\hltinyboxbackslash=\hbox{\begin{tiny}\verb.\.\end{tiny}}%
\setbox\hltinyboxgreaterthan=\hbox{\begin{tiny}\verb.>.\end{tiny}}%
\setbox\hltinyboxpercent=\hbox{\begin{tiny}\verb.%.\end{tiny}}%
\setbox\hltinyboxhat=\hbox{\begin{tiny}\verb.^.\end{tiny}}%
\setbox\hltinyboxsinglequote=\hbox{\begin{tiny}\verb.'.\end{tiny}}%
\setbox\hltinyboxbacktick=\hbox{\begin{tiny}\verb.`.\end{tiny}}%
\setbox\hltinyboxhat=\hbox{\begin{tiny}\verb.^.\end{tiny}}%



\newsavebox{\hlscriptsizeboxclosebrace}%
\newsavebox{\hlscriptsizeboxopenbrace}%
\newsavebox{\hlscriptsizeboxbackslash}%
\newsavebox{\hlscriptsizeboxlessthan}%
\newsavebox{\hlscriptsizeboxgreaterthan}%
\newsavebox{\hlscriptsizeboxdollar}%
\newsavebox{\hlscriptsizeboxunderscore}%
\newsavebox{\hlscriptsizeboxand}%
\newsavebox{\hlscriptsizeboxhash}%
\newsavebox{\hlscriptsizeboxat}%
\newsavebox{\hlscriptsizeboxpercent}% 
\newsavebox{\hlscriptsizeboxhat}%
\newsavebox{\hlscriptsizeboxsinglequote}%
\newsavebox{\hlscriptsizeboxbacktick}%

\setbox\hlscriptsizeboxopenbrace=\hbox{\begin{scriptsize}\verb.{.\end{scriptsize}}%
\setbox\hlscriptsizeboxclosebrace=\hbox{\begin{scriptsize}\verb.}.\end{scriptsize}}%
\setbox\hlscriptsizeboxlessthan=\hbox{\begin{scriptsize}\verb.<.\end{scriptsize}}%
\setbox\hlscriptsizeboxdollar=\hbox{\begin{scriptsize}\verb.$.\end{scriptsize}}%
\setbox\hlscriptsizeboxunderscore=\hbox{\begin{scriptsize}\verb._.\end{scriptsize}}%
\setbox\hlscriptsizeboxand=\hbox{\begin{scriptsize}\verb.&.\end{scriptsize}}%
\setbox\hlscriptsizeboxhash=\hbox{\begin{scriptsize}\verb.#.\end{scriptsize}}%
\setbox\hlscriptsizeboxat=\hbox{\begin{scriptsize}\verb.@.\end{scriptsize}}%
\setbox\hlscriptsizeboxbackslash=\hbox{\begin{scriptsize}\verb.\.\end{scriptsize}}%
\setbox\hlscriptsizeboxgreaterthan=\hbox{\begin{scriptsize}\verb.>.\end{scriptsize}}%
\setbox\hlscriptsizeboxpercent=\hbox{\begin{scriptsize}\verb.%.\end{scriptsize}}%
\setbox\hlscriptsizeboxhat=\hbox{\begin{scriptsize}\verb.^.\end{scriptsize}}%
\setbox\hlscriptsizeboxsinglequote=\hbox{\begin{scriptsize}\verb.'.\end{scriptsize}}%
\setbox\hlscriptsizeboxbacktick=\hbox{\begin{scriptsize}\verb.`.\end{scriptsize}}%
\setbox\hlscriptsizeboxhat=\hbox{\begin{scriptsize}\verb.^.\end{scriptsize}}%



\newsavebox{\hlfootnotesizeboxclosebrace}%
\newsavebox{\hlfootnotesizeboxopenbrace}%
\newsavebox{\hlfootnotesizeboxbackslash}%
\newsavebox{\hlfootnotesizeboxlessthan}%
\newsavebox{\hlfootnotesizeboxgreaterthan}%
\newsavebox{\hlfootnotesizeboxdollar}%
\newsavebox{\hlfootnotesizeboxunderscore}%
\newsavebox{\hlfootnotesizeboxand}%
\newsavebox{\hlfootnotesizeboxhash}%
\newsavebox{\hlfootnotesizeboxat}%
\newsavebox{\hlfootnotesizeboxpercent}% 
\newsavebox{\hlfootnotesizeboxhat}%
\newsavebox{\hlfootnotesizeboxsinglequote}%
\newsavebox{\hlfootnotesizeboxbacktick}%

\setbox\hlfootnotesizeboxopenbrace=\hbox{\begin{footnotesize}\verb.{.\end{footnotesize}}%
\setbox\hlfootnotesizeboxclosebrace=\hbox{\begin{footnotesize}\verb.}.\end{footnotesize}}%
\setbox\hlfootnotesizeboxlessthan=\hbox{\begin{footnotesize}\verb.<.\end{footnotesize}}%
\setbox\hlfootnotesizeboxdollar=\hbox{\begin{footnotesize}\verb.$.\end{footnotesize}}%
\setbox\hlfootnotesizeboxunderscore=\hbox{\begin{footnotesize}\verb._.\end{footnotesize}}%
\setbox\hlfootnotesizeboxand=\hbox{\begin{footnotesize}\verb.&.\end{footnotesize}}%
\setbox\hlfootnotesizeboxhash=\hbox{\begin{footnotesize}\verb.#.\end{footnotesize}}%
\setbox\hlfootnotesizeboxat=\hbox{\begin{footnotesize}\verb.@.\end{footnotesize}}%
\setbox\hlfootnotesizeboxbackslash=\hbox{\begin{footnotesize}\verb.\.\end{footnotesize}}%
\setbox\hlfootnotesizeboxgreaterthan=\hbox{\begin{footnotesize}\verb.>.\end{footnotesize}}%
\setbox\hlfootnotesizeboxpercent=\hbox{\begin{footnotesize}\verb.%.\end{footnotesize}}%
\setbox\hlfootnotesizeboxhat=\hbox{\begin{footnotesize}\verb.^.\end{footnotesize}}%
\setbox\hlfootnotesizeboxsinglequote=\hbox{\begin{footnotesize}\verb.'.\end{footnotesize}}%
\setbox\hlfootnotesizeboxbacktick=\hbox{\begin{footnotesize}\verb.`.\end{footnotesize}}%
\setbox\hlfootnotesizeboxhat=\hbox{\begin{footnotesize}\verb.^.\end{footnotesize}}%



\newsavebox{\hlsmallboxclosebrace}%
\newsavebox{\hlsmallboxopenbrace}%
\newsavebox{\hlsmallboxbackslash}%
\newsavebox{\hlsmallboxlessthan}%
\newsavebox{\hlsmallboxgreaterthan}%
\newsavebox{\hlsmallboxdollar}%
\newsavebox{\hlsmallboxunderscore}%
\newsavebox{\hlsmallboxand}%
\newsavebox{\hlsmallboxhash}%
\newsavebox{\hlsmallboxat}%
\newsavebox{\hlsmallboxpercent}% 
\newsavebox{\hlsmallboxhat}%
\newsavebox{\hlsmallboxsinglequote}%
\newsavebox{\hlsmallboxbacktick}%

\setbox\hlsmallboxopenbrace=\hbox{\begin{small}\verb.{.\end{small}}%
\setbox\hlsmallboxclosebrace=\hbox{\begin{small}\verb.}.\end{small}}%
\setbox\hlsmallboxlessthan=\hbox{\begin{small}\verb.<.\end{small}}%
\setbox\hlsmallboxdollar=\hbox{\begin{small}\verb.$.\end{small}}%
\setbox\hlsmallboxunderscore=\hbox{\begin{small}\verb._.\end{small}}%
\setbox\hlsmallboxand=\hbox{\begin{small}\verb.&.\end{small}}%
\setbox\hlsmallboxhash=\hbox{\begin{small}\verb.#.\end{small}}%
\setbox\hlsmallboxat=\hbox{\begin{small}\verb.@.\end{small}}%
\setbox\hlsmallboxbackslash=\hbox{\begin{small}\verb.\.\end{small}}%
\setbox\hlsmallboxgreaterthan=\hbox{\begin{small}\verb.>.\end{small}}%
\setbox\hlsmallboxpercent=\hbox{\begin{small}\verb.%.\end{small}}%
\setbox\hlsmallboxhat=\hbox{\begin{small}\verb.^.\end{small}}%
\setbox\hlsmallboxsinglequote=\hbox{\begin{small}\verb.'.\end{small}}%
\setbox\hlsmallboxbacktick=\hbox{\begin{small}\verb.`.\end{small}}%
\setbox\hlsmallboxhat=\hbox{\begin{small}\verb.^.\end{small}}%



\newsavebox{\hllargeboxclosebrace}%
\newsavebox{\hllargeboxopenbrace}%
\newsavebox{\hllargeboxbackslash}%
\newsavebox{\hllargeboxlessthan}%
\newsavebox{\hllargeboxgreaterthan}%
\newsavebox{\hllargeboxdollar}%
\newsavebox{\hllargeboxunderscore}%
\newsavebox{\hllargeboxand}%
\newsavebox{\hllargeboxhash}%
\newsavebox{\hllargeboxat}%
\newsavebox{\hllargeboxpercent}% 
\newsavebox{\hllargeboxhat}%
\newsavebox{\hllargeboxsinglequote}%
\newsavebox{\hllargeboxbacktick}%

\setbox\hllargeboxopenbrace=\hbox{\begin{large}\verb.{.\end{large}}%
\setbox\hllargeboxclosebrace=\hbox{\begin{large}\verb.}.\end{large}}%
\setbox\hllargeboxlessthan=\hbox{\begin{large}\verb.<.\end{large}}%
\setbox\hllargeboxdollar=\hbox{\begin{large}\verb.$.\end{large}}%
\setbox\hllargeboxunderscore=\hbox{\begin{large}\verb._.\end{large}}%
\setbox\hllargeboxand=\hbox{\begin{large}\verb.&.\end{large}}%
\setbox\hllargeboxhash=\hbox{\begin{large}\verb.#.\end{large}}%
\setbox\hllargeboxat=\hbox{\begin{large}\verb.@.\end{large}}%
\setbox\hllargeboxbackslash=\hbox{\begin{large}\verb.\.\end{large}}%
\setbox\hllargeboxgreaterthan=\hbox{\begin{large}\verb.>.\end{large}}%
\setbox\hllargeboxpercent=\hbox{\begin{large}\verb.%.\end{large}}%
\setbox\hllargeboxhat=\hbox{\begin{large}\verb.^.\end{large}}%
\setbox\hllargeboxsinglequote=\hbox{\begin{large}\verb.'.\end{large}}%
\setbox\hllargeboxbacktick=\hbox{\begin{large}\verb.`.\end{large}}%
\setbox\hllargeboxhat=\hbox{\begin{large}\verb.^.\end{large}}%



\newsavebox{\hlLargeboxclosebrace}%
\newsavebox{\hlLargeboxopenbrace}%
\newsavebox{\hlLargeboxbackslash}%
\newsavebox{\hlLargeboxlessthan}%
\newsavebox{\hlLargeboxgreaterthan}%
\newsavebox{\hlLargeboxdollar}%
\newsavebox{\hlLargeboxunderscore}%
\newsavebox{\hlLargeboxand}%
\newsavebox{\hlLargeboxhash}%
\newsavebox{\hlLargeboxat}%
\newsavebox{\hlLargeboxpercent}% 
\newsavebox{\hlLargeboxhat}%
\newsavebox{\hlLargeboxsinglequote}%
\newsavebox{\hlLargeboxbacktick}%

\setbox\hlLargeboxopenbrace=\hbox{\begin{Large}\verb.{.\end{Large}}%
\setbox\hlLargeboxclosebrace=\hbox{\begin{Large}\verb.}.\end{Large}}%
\setbox\hlLargeboxlessthan=\hbox{\begin{Large}\verb.<.\end{Large}}%
\setbox\hlLargeboxdollar=\hbox{\begin{Large}\verb.$.\end{Large}}%
\setbox\hlLargeboxunderscore=\hbox{\begin{Large}\verb._.\end{Large}}%
\setbox\hlLargeboxand=\hbox{\begin{Large}\verb.&.\end{Large}}%
\setbox\hlLargeboxhash=\hbox{\begin{Large}\verb.#.\end{Large}}%
\setbox\hlLargeboxat=\hbox{\begin{Large}\verb.@.\end{Large}}%
\setbox\hlLargeboxbackslash=\hbox{\begin{Large}\verb.\.\end{Large}}%
\setbox\hlLargeboxgreaterthan=\hbox{\begin{Large}\verb.>.\end{Large}}%
\setbox\hlLargeboxpercent=\hbox{\begin{Large}\verb.%.\end{Large}}%
\setbox\hlLargeboxhat=\hbox{\begin{Large}\verb.^.\end{Large}}%
\setbox\hlLargeboxsinglequote=\hbox{\begin{Large}\verb.'.\end{Large}}%
\setbox\hlLargeboxbacktick=\hbox{\begin{Large}\verb.`.\end{Large}}%
\setbox\hlLargeboxhat=\hbox{\begin{Large}\verb.^.\end{Large}}%



\newsavebox{\hlLARGEboxclosebrace}%
\newsavebox{\hlLARGEboxopenbrace}%
\newsavebox{\hlLARGEboxbackslash}%
\newsavebox{\hlLARGEboxlessthan}%
\newsavebox{\hlLARGEboxgreaterthan}%
\newsavebox{\hlLARGEboxdollar}%
\newsavebox{\hlLARGEboxunderscore}%
\newsavebox{\hlLARGEboxand}%
\newsavebox{\hlLARGEboxhash}%
\newsavebox{\hlLARGEboxat}%
\newsavebox{\hlLARGEboxpercent}% 
\newsavebox{\hlLARGEboxhat}%
\newsavebox{\hlLARGEboxsinglequote}%
\newsavebox{\hlLARGEboxbacktick}%

\setbox\hlLARGEboxopenbrace=\hbox{\begin{LARGE}\verb.{.\end{LARGE}}%
\setbox\hlLARGEboxclosebrace=\hbox{\begin{LARGE}\verb.}.\end{LARGE}}%
\setbox\hlLARGEboxlessthan=\hbox{\begin{LARGE}\verb.<.\end{LARGE}}%
\setbox\hlLARGEboxdollar=\hbox{\begin{LARGE}\verb.$.\end{LARGE}}%
\setbox\hlLARGEboxunderscore=\hbox{\begin{LARGE}\verb._.\end{LARGE}}%
\setbox\hlLARGEboxand=\hbox{\begin{LARGE}\verb.&.\end{LARGE}}%
\setbox\hlLARGEboxhash=\hbox{\begin{LARGE}\verb.#.\end{LARGE}}%
\setbox\hlLARGEboxat=\hbox{\begin{LARGE}\verb.@.\end{LARGE}}%
\setbox\hlLARGEboxbackslash=\hbox{\begin{LARGE}\verb.\.\end{LARGE}}%
\setbox\hlLARGEboxgreaterthan=\hbox{\begin{LARGE}\verb.>.\end{LARGE}}%
\setbox\hlLARGEboxpercent=\hbox{\begin{LARGE}\verb.%.\end{LARGE}}%
\setbox\hlLARGEboxhat=\hbox{\begin{LARGE}\verb.^.\end{LARGE}}%
\setbox\hlLARGEboxsinglequote=\hbox{\begin{LARGE}\verb.'.\end{LARGE}}%
\setbox\hlLARGEboxbacktick=\hbox{\begin{LARGE}\verb.`.\end{LARGE}}%
\setbox\hlLARGEboxhat=\hbox{\begin{LARGE}\verb.^.\end{LARGE}}%



\newsavebox{\hlhugeboxclosebrace}%
\newsavebox{\hlhugeboxopenbrace}%
\newsavebox{\hlhugeboxbackslash}%
\newsavebox{\hlhugeboxlessthan}%
\newsavebox{\hlhugeboxgreaterthan}%
\newsavebox{\hlhugeboxdollar}%
\newsavebox{\hlhugeboxunderscore}%
\newsavebox{\hlhugeboxand}%
\newsavebox{\hlhugeboxhash}%
\newsavebox{\hlhugeboxat}%
\newsavebox{\hlhugeboxpercent}% 
\newsavebox{\hlhugeboxhat}%
\newsavebox{\hlhugeboxsinglequote}%
\newsavebox{\hlhugeboxbacktick}%

\setbox\hlhugeboxopenbrace=\hbox{\begin{huge}\verb.{.\end{huge}}%
\setbox\hlhugeboxclosebrace=\hbox{\begin{huge}\verb.}.\end{huge}}%
\setbox\hlhugeboxlessthan=\hbox{\begin{huge}\verb.<.\end{huge}}%
\setbox\hlhugeboxdollar=\hbox{\begin{huge}\verb.$.\end{huge}}%
\setbox\hlhugeboxunderscore=\hbox{\begin{huge}\verb._.\end{huge}}%
\setbox\hlhugeboxand=\hbox{\begin{huge}\verb.&.\end{huge}}%
\setbox\hlhugeboxhash=\hbox{\begin{huge}\verb.#.\end{huge}}%
\setbox\hlhugeboxat=\hbox{\begin{huge}\verb.@.\end{huge}}%
\setbox\hlhugeboxbackslash=\hbox{\begin{huge}\verb.\.\end{huge}}%
\setbox\hlhugeboxgreaterthan=\hbox{\begin{huge}\verb.>.\end{huge}}%
\setbox\hlhugeboxpercent=\hbox{\begin{huge}\verb.%.\end{huge}}%
\setbox\hlhugeboxhat=\hbox{\begin{huge}\verb.^.\end{huge}}%
\setbox\hlhugeboxsinglequote=\hbox{\begin{huge}\verb.'.\end{huge}}%
\setbox\hlhugeboxbacktick=\hbox{\begin{huge}\verb.`.\end{huge}}%
\setbox\hlhugeboxhat=\hbox{\begin{huge}\verb.^.\end{huge}}%



\newsavebox{\hlHugeboxclosebrace}%
\newsavebox{\hlHugeboxopenbrace}%
\newsavebox{\hlHugeboxbackslash}%
\newsavebox{\hlHugeboxlessthan}%
\newsavebox{\hlHugeboxgreaterthan}%
\newsavebox{\hlHugeboxdollar}%
\newsavebox{\hlHugeboxunderscore}%
\newsavebox{\hlHugeboxand}%
\newsavebox{\hlHugeboxhash}%
\newsavebox{\hlHugeboxat}%
\newsavebox{\hlHugeboxpercent}% 
\newsavebox{\hlHugeboxhat}%
\newsavebox{\hlHugeboxsinglequote}%
\newsavebox{\hlHugeboxbacktick}%

\setbox\hlHugeboxopenbrace=\hbox{\begin{Huge}\verb.{.\end{Huge}}%
\setbox\hlHugeboxclosebrace=\hbox{\begin{Huge}\verb.}.\end{Huge}}%
\setbox\hlHugeboxlessthan=\hbox{\begin{Huge}\verb.<.\end{Huge}}%
\setbox\hlHugeboxdollar=\hbox{\begin{Huge}\verb.$.\end{Huge}}%
\setbox\hlHugeboxunderscore=\hbox{\begin{Huge}\verb._.\end{Huge}}%
\setbox\hlHugeboxand=\hbox{\begin{Huge}\verb.&.\end{Huge}}%
\setbox\hlHugeboxhash=\hbox{\begin{Huge}\verb.#.\end{Huge}}%
\setbox\hlHugeboxat=\hbox{\begin{Huge}\verb.@.\end{Huge}}%
\setbox\hlHugeboxbackslash=\hbox{\begin{Huge}\verb.\.\end{Huge}}%
\setbox\hlHugeboxgreaterthan=\hbox{\begin{Huge}\verb.>.\end{Huge}}%
\setbox\hlHugeboxpercent=\hbox{\begin{Huge}\verb.%.\end{Huge}}%
\setbox\hlHugeboxhat=\hbox{\begin{Huge}\verb.^.\end{Huge}}%
\setbox\hlHugeboxsinglequote=\hbox{\begin{Huge}\verb.'.\end{Huge}}%
\setbox\hlHugeboxbacktick=\hbox{\begin{Huge}\verb.`.\end{Huge}}%
\setbox\hlHugeboxhat=\hbox{\begin{Huge}\verb.^.\end{Huge}}%
 

\def\urltilda{\kern -.15em\lower .7ex\hbox{\~{}}\kern .04em}%

\newcommand{\hlstd}[1]{\textcolor[rgb]{0,0,0}{#1}}%
\newcommand{\hlnum}[1]{\textcolor[rgb]{0.16,0.16,1}{#1}}
\newcommand{\hlesc}[1]{\textcolor[rgb]{1,0,1}{#1}}
\newcommand{\hlstr}[1]{\textcolor[rgb]{1,0,0}{#1}}
\newcommand{\hldstr}[1]{\textcolor[rgb]{0.51,0.51,0}{#1}}
\newcommand{\hlslc}[1]{\textcolor[rgb]{0.51,0.51,0.51}{\it{#1}}}
\newcommand{\hlcom}[1]{\textcolor[rgb]{0.51,0.51,0.51}{\it{#1}}}
\newcommand{\hldir}[1]{\textcolor[rgb]{0,0.51,0}{#1}}
\newcommand{\hlsym}[1]{\textcolor[rgb]{0,0,0}{#1}}
\newcommand{\hlline}[1]{\textcolor[rgb]{0.33,0.33,0.33}{#1}}
\newcommand{\hlkwa}[1]{\textcolor[rgb]{0,0,0}{\bf{#1}}}
\newcommand{\hlkwb}[1]{\textcolor[rgb]{0.51,0,0}{#1}}
\newcommand{\hlkwc}[1]{\textcolor[rgb]{0,0,0}{\bf{#1}}}
\newcommand{\hlkwd}[1]{\textcolor[rgb]{0,0,0.51}{#1}}

\newcommand{\hlnumber}[1]{\textcolor[rgb]{0.0823529411764706,0.0784313725490196,0.709803921568627}{#1}}%
\newcommand{\hlfunctioncall}[1]{\textcolor[rgb]{1,0,0}{#1}}%
\newcommand{\hlstring}[1]{\textcolor[rgb]{0.6,0.6,1}{#1}}%
\newcommand{\hlkeyword}[1]{\textcolor[rgb]{0,0,0}{\textbf{#1}}}%
\newcommand{\hlargument}[1]{\textcolor[rgb]{0.694117647058824,0.247058823529412,0.0196078431372549}{#1}}%
\newcommand{\hlcomment}[1]{\textcolor[rgb]{0.8,0.8,0.8}{#1}}%
\newcommand{\hlroxygencomment}[1]{\textcolor[rgb]{0,0.592156862745098,1}{#1}}%
\newcommand{\hlformalargs}[1]{\textcolor[rgb]{0.0705882352941176,0.713725490196078,0.0705882352941176}{#1}}%
\newcommand{\hleqformalargs}[1]{\textcolor[rgb]{0.0705882352941176,0.713725490196078,0.0705882352941176}{#1}}%
\newcommand{\hlassignement}[1]{\textcolor[rgb]{0.215686274509804,0.215686274509804,0.384313725490196}{\textbf{#1}}}%
\newcommand{\hlpackage}[1]{\textcolor[rgb]{0.588235294117647,0.713725490196078,0.145098039215686}{#1}}%
\newcommand{\hlslot}[1]{\textit{#1}}%
\newcommand{\hlsymbol}[1]{\textcolor[rgb]{0,0,0}{#1}}%
\newcommand{\hlprompt}[1]{\textcolor[rgb]{0,0,0}{#1}}%

\usepackage{graphicx}
\usepackage{fancyhdr}
\usepackage[T1]{fontenc}
\usepackage{lmodern}
%\usepackage[francais]{babel}
%\usepackage[frenchb]{babel}
%\usepackage[applemac]{inputenc}
\usepackage{lipsum}
\usepackage{vmargin}
\usepackage{lastpage}
\usepackage{enumerate}
\usepackage{pdfpages}

\usepackage{lscape}
\usepackage{color}
\usepackage{listings}
\usepackage{verbatim}
\usepackage{amssymb, amsmath}
\usepackage{wrapfig}
\usepackage[colorlinks=true,linkcolor=blue,citecolor=red,bookmarks=true]{hyperref} % Hyperlink capabilities, should load last
\usepackage[all]{hypcap}
\usepackage{atbegshi}
\usepackage{tikz}


\begin{document}

\begin{titlepage}
\vspace*{3cm}
\begin{center}

\huge{\bf Selectivity functions that estimate the relation between a variable and its probabilities}\\

\vspace*{2cm}
\large{By Benoit Bruneau}
\end{center}
\vspace*{4cm}

%\hangindent=1cm
%\begin{flushleft}
\begin{description}
\item[Package:] `bmisc'
\item[Version:] 0.2-12
\item[Depends:] car, lattice, zoo, robustbase, methods, and tcltk
\item[Author \& Maintainer:] Benoit Bruneau (\href{mailto:benoit.bruneau1@gmail.com}{benoit.bruneau1@gmail.com})
\item[Description:] These functions can be used to estimate probabilities \verb=[0,1]= by specifying the inflection points of a relation. Described relations are of type `const', `full', `ramp' and `logit'.
\item[License:] LGPL $\geqslant$ 3.0
\end{description}


\vspace*{\fill}


\end{titlepage}

\tableofcontents
\newpage

\section{Types `const', `full', and `plat.full'}
\noindent These relations are the simplest that can be used. While `const' stands for a constant probability of one for all 
values of \verb#x# (Figure~\ref{fig1}), the other two have "all-or-nothing" types of probabilities. One or two thresholds 
(inflection points) need to be defined for types `full' and `plat.full'. The main difference between `full'
(Figure~\ref{fig2} \& \ref{fig3} ) 
and `plat.full' (Figure~\ref{fig4}, \ref{fig5} \& \ref{fig6}) types are the number of thresholds. For all types, `plat' stands for "plateau".\\*

\begin{description}
\item[const.sel]\verb#(x)#
\item[full.sel]\verb#(x, infl1, pos=TRUE, lv=0, uv=1)#
\item[plat.full.sel]\verb#(x, infl1, infl2, pos=TRUE, lv=c(0,0), uv=c(1,1))#
\end{description}
where \verb#x# is a numeric vector for which probabilities are estimated, \verb#infl1# and \verb#infl2# are the inflection points, 
\verb#pos# indicates if the trend at the beginning is positive  (\verb#TRUE#) or negative (\verb#FALSE#), \verb#lv# defines the 
lower probability values of the relation, and \verb#uv# defines the upper probability values of the relation. By default, 
all fonctions have \verb#pos=TRUE#, \verb#lv=c(0,0)#, and \verb#uv=c(1,1)#.\\*

Here is an example for `const' type:


\begin{Hchunk}
\begin{normalsize}
\begin{Hinput}
\ttfamily\noindent
\hlprompt{\usebox{\hlnormalsizeboxgreaterthan}{\ }}\hlsymbol{data}\hlassignement{=}\hlnumber{0}\hlkeyword{:}\hlnumber{3000}\mbox{}
\normalfont
\end{Hinput}


\begin{Hinput}
\ttfamily\noindent
\hlprompt{\usebox{\hlnormalsizeboxgreaterthan}{\ }}\hlfunctioncall{const.sel}\hlkeyword{(}\hlargument{x}\hlargument{=}\hlsymbol{data}\hlkeyword{)}\mbox{}
\normalfont
\end{Hinput}


\end{normalsize}
\end{Hchunk}


\begin{figure}[h]
\vspace{-20pt}
\begin{center}
\includegraphics{selectivity-003}
\end{center}
  \vspace{-30pt}
  \caption{Type `const' probabilities.}
  \vspace{-10pt}
\label{fig1}
\end{figure}
\vspace*{\fill}

%%%%%%%%%%%%%%%%%%%%%%%%%%%%%%%%%%%%%%%%%%%%%%%%%%%%%%%%%%%%%%%%%%%%%%%%%%%%%%%%%%%%%%%%%%%%%%%%%%%

\newpage

Here are examples for `full' type:
\begin{Hchunk}
\begin{normalsize}
\begin{Hinput}
\ttfamily\noindent
\hlprompt{\usebox{\hlnormalsizeboxgreaterthan}{\ }}\hlsymbol{data}\hlassignement{=}\hlnumber{0}\hlkeyword{:}\hlnumber{3000}\mbox{}
\normalfont
\end{Hinput}


\begin{Hinput}
\ttfamily\noindent
\hlprompt{\usebox{\hlnormalsizeboxgreaterthan}{\ }}\hlfunctioncall{full.sel}\hlkeyword{(}\hlargument{x}\hlargument{=}\hlsymbol{data}\hlkeyword{,}{\ }\hlargument{infl1}\hlargument{=}\hlnumber{1500}\hlkeyword{,}{\ }\hlargument{pos}\hlargument{=}\hlnumber{TRUE}\hlkeyword{)}\mbox{}
\normalfont
\end{Hinput}


\begin{Hinput}
\ttfamily\noindent
\hlprompt{\usebox{\hlnormalsizeboxgreaterthan}{\ }}\hlfunctioncall{full.sel}\hlkeyword{(}\hlargument{x}\hlargument{=}\hlsymbol{data}\hlkeyword{,}{\ }\hlargument{infl1}\hlargument{=}\hlnumber{1500}\hlkeyword{,}{\ }\hlargument{pos}\hlargument{=}\hlnumber{FALSE}\hlkeyword{)}\mbox{}
\normalfont
\end{Hinput}


\end{normalsize}
\end{Hchunk}


\begin{figure}[h]
\vspace{-20pt}
\begin{center}
\includegraphics{selectivity-005}
\end{center}
\vspace{-30pt}
\caption{Type `full' probabilities (left -> pos=TRUE |  right -> pos=FALSE).}
\vspace{-10pt}
\label{fig2}
\end{figure}

\begin{Hchunk}
\begin{normalsize}
\begin{Hinput}
\ttfamily\noindent
\hlprompt{\usebox{\hlnormalsizeboxgreaterthan}{\ }}\hlsymbol{data}\hlassignement{=}\hlnumber{0}\hlkeyword{:}\hlnumber{3000}\mbox{}
\normalfont
\end{Hinput}


\begin{Hinput}
\ttfamily\noindent
\hlprompt{\usebox{\hlnormalsizeboxgreaterthan}{\ }}\hlfunctioncall{full.sel}\hlkeyword{(}\hlargument{x}\hlargument{=}\hlsymbol{data}\hlkeyword{,}{\ }\hlargument{infl1}\hlargument{=}\hlnumber{1500}\hlkeyword{,}{\ }\hlargument{pos}\hlargument{=}\hlnumber{TRUE}\hlkeyword{,}{\ }\hlargument{lv}\hlargument{=}\hlnumber{0.2}\hlkeyword{,}\hlargument{uv}\hlargument{=}\hlnumber{0.8}\hlkeyword{)}\mbox{}
\normalfont
\end{Hinput}


\begin{Hinput}
\ttfamily\noindent
\hlprompt{\usebox{\hlnormalsizeboxgreaterthan}{\ }}\hlfunctioncall{full.sel}\hlkeyword{(}\hlargument{x}\hlargument{=}\hlsymbol{data}\hlkeyword{,}{\ }\hlargument{infl1}\hlargument{=}\hlnumber{1500}\hlkeyword{,}{\ }\hlargument{pos}\hlargument{=}\hlnumber{FALSE}\hlkeyword{,}{\ }\hlargument{lv}\hlargument{=}\hlnumber{0.2}\hlkeyword{,}\hlargument{uv}\hlargument{=}\hlnumber{0.8}\hlkeyword{)}\mbox{}
\normalfont
\end{Hinput}


\end{normalsize}
\end{Hchunk}


\begin{figure}[h]
\vspace{-20pt}
\begin{center}
\includegraphics{selectivity-007}
\end{center}
\vspace{-30pt}
\caption{Type `full' probabilities (left -> pos=TRUE |  right -> pos=FALSE). In this example, minimum and maximum probabilities are respectively lv=0.2 and uv=0.8.}
\vspace{-10pt}
\label{fig3}
\end{figure}

%%%%%%%%%%%%%%%%%%%%%%%%%%%%%%%%%%%%%%%%%%%%%%%%%%%%%%%%%%%%%%%%%%%%%%%%%%%%%%%%%%%%%%%%%%%%%%%%%%%
\newpage

Here are examples for `plat.full' type:
\begin{Hchunk}
\begin{normalsize}
\begin{Hinput}
\ttfamily\noindent
\hlprompt{\usebox{\hlnormalsizeboxgreaterthan}{\ }}\hlsymbol{data}\hlassignement{=}\hlnumber{0}\hlkeyword{:}\hlnumber{3000}\mbox{}
\normalfont
\end{Hinput}


\begin{Hinput}
\ttfamily\noindent
\hlprompt{\usebox{\hlnormalsizeboxgreaterthan}{\ }}\hlfunctioncall{plat.full.sel}\hlkeyword{(}\hlargument{x}\hlargument{=}\hlsymbol{data}\hlkeyword{,}{\ }\hlargument{infl1}\hlargument{=}\hlnumber{1000}\hlkeyword{,}{\ }\hlargument{infl2}\hlargument{=}\hlnumber{2000}\hlkeyword{,}{\ }\hlargument{pos}\hlargument{=}\hlnumber{TRUE}\hlkeyword{)}\mbox{}
\normalfont
\end{Hinput}


\begin{Hinput}
\ttfamily\noindent
\hlprompt{\usebox{\hlnormalsizeboxgreaterthan}{\ }}\hlfunctioncall{plat.full.sel}\hlkeyword{(}\hlargument{x}\hlargument{=}\hlsymbol{data}\hlkeyword{,}{\ }\hlargument{infl1}\hlargument{=}\hlnumber{1000}\hlkeyword{,}{\ }\hlargument{infl2}\hlargument{=}\hlnumber{2000}\hlkeyword{,}{\ }\hlargument{pos}\hlargument{=}\hlnumber{FALSE}\hlkeyword{)}\mbox{}
\normalfont
\end{Hinput}


\end{normalsize}
\end{Hchunk}

\begin{figure}[h]
\vspace{-20pt}
\begin{center}
\includegraphics{selectivity-009}
\end{center}
\vspace{-30pt}
\caption{Type `plat.full' probabilities (left -> pos=TRUE |  right -> pos=FALSE).}
\vspace{-10pt}
\label{fig4}
\end{figure}



\begin{Hchunk}
\begin{normalsize}
\begin{Hinput}
\ttfamily\noindent
\hlprompt{\usebox{\hlnormalsizeboxgreaterthan}{\ }}\hlsymbol{data}\hlassignement{=}\hlnumber{0}\hlkeyword{:}\hlnumber{3000}\mbox{}
\normalfont
\end{Hinput}


\begin{Hinput}
\ttfamily\noindent
\hlprompt{\usebox{\hlnormalsizeboxgreaterthan}{\ }}\hlfunctioncall{plat.full.sel}\hlkeyword{(}\hlargument{x}\hlargument{=}\hlsymbol{data}\hlkeyword{,}{\ }\hlargument{infl1}\hlargument{=}\hlnumber{1000}\hlkeyword{,}{\ }\hlargument{infl2}\hlargument{=}\hlnumber{2000}\hlkeyword{,}{\ }\hlargument{pos}\hlargument{=}\hlnumber{TRUE}\hlkeyword{,}{\ }\hlargument{lv}\hlargument{=}\hlnumber{0.2}\hlkeyword{,}\hlargument{uv}\hlargument{=}\hlnumber{0.8}\hlkeyword{)}\mbox{}
\normalfont
\end{Hinput}


\begin{Hinput}
\ttfamily\noindent
\hlprompt{\usebox{\hlnormalsizeboxgreaterthan}{\ }}\hlfunctioncall{plat.full.sel}\hlkeyword{(}\hlargument{x}\hlargument{=}\hlsymbol{data}\hlkeyword{,}{\ }\hlargument{infl1}\hlargument{=}\hlnumber{1000}\hlkeyword{,}{\ }\hlargument{infl2}\hlargument{=}\hlnumber{2000}\hlkeyword{,}{\ }\hlargument{pos}\hlargument{=}\hlnumber{FALSE}\hlkeyword{,}{\ }\hlargument{lv}\hlargument{=}\hlnumber{0.2}\hlkeyword{,}\hlargument{uv}\hlargument{=}\hlnumber{0.8}\hlkeyword{)}\mbox{}
\normalfont
\end{Hinput}


\end{normalsize}
\end{Hchunk}

\begin{figure}[h]
\vspace{-20pt}
\begin{center}
\includegraphics{selectivity-011}
\end{center}
\vspace{-30pt}
\caption{Type `plat.full' probabilities (left -> pos=TRUE |  right -> pos=FALSE). In this example, minimum and maximum probabilities are respectively lv=0.2 and uv=0.8.}
\vspace{-10pt}
\label{fig5}
\end{figure}

\begin{Hchunk}
\begin{normalsize}
\begin{Hinput}
\ttfamily\noindent
\hlprompt{\usebox{\hlnormalsizeboxgreaterthan}{\ }}\hlsymbol{data}\hlassignement{=}\hlnumber{0}\hlkeyword{:}\hlnumber{3000}\mbox{}
\normalfont
\end{Hinput}


\begin{Hinput}
\ttfamily\noindent
\hlprompt{\usebox{\hlnormalsizeboxgreaterthan}{\ }}\hlfunctioncall{plat.full.sel}\hlkeyword{(}\hlargument{x}\hlargument{=}\hlsymbol{data}\hlkeyword{,}{\ }\hlargument{infl1}\hlargument{=}\hlnumber{1000}\hlkeyword{,}{\ }\hlargument{infl2}\hlargument{=}\hlnumber{2000}\hlkeyword{,}{\ }\hlargument{pos}\hlargument{=}\hlnumber{TRUE}\hlkeyword{,}{\ }\hlargument{lv}\hlargument{=}\hlfunctioncall{c}\hlkeyword{(}\hlnumber{0.2}\hlkeyword{,}\hlnumber{0.4}\hlkeyword{)}\hlkeyword{,}\hlargument{uv}\hlargument{=}\hlnumber{0.8}\hlkeyword{)}\mbox{}
\normalfont
\end{Hinput}


\begin{Hinput}
\ttfamily\noindent
\hlprompt{\usebox{\hlnormalsizeboxgreaterthan}{\ }}\hlfunctioncall{plat.full.sel}\hlkeyword{(}\hlargument{x}\hlargument{=}\hlsymbol{data}\hlkeyword{,}{\ }\hlargument{infl1}\hlargument{=}\hlnumber{1000}\hlkeyword{,}{\ }\hlargument{infl2}\hlargument{=}\hlnumber{2000}\hlkeyword{,}{\ }\hlargument{pos}\hlargument{=}\hlnumber{FALSE}\hlkeyword{,}{\ }\hlargument{lv}\hlargument{=}\hlnumber{0.2}\hlkeyword{,}\hlargument{uv}\hlargument{=}\hlfunctioncall{c}\hlkeyword{(}\hlnumber{0.8}\hlkeyword{,}\hlnumber{0.6}\hlkeyword{)}\hlkeyword{)}\mbox{}
\normalfont
\end{Hinput}


\end{normalsize}
\end{Hchunk}

\begin{figure}[h]
\vspace{-20pt}
\begin{center}
\includegraphics{selectivity-013}
\end{center}
\vspace{-30pt}
\caption{Type `plat.full' probabilities (left -> pos=TRUE, lv=c(0.2,0.4) , uv=0.8 | right -> pos=FALSE, lv=0.2, uv=c(0.8,0.6)).}
\vspace{-10pt}
\label{fig6}
\end{figure}


%%%%%%%%%%%%%%%%%%%%%%%%%%%%%%%%%%%%%%%%%%%%%%%%%%%%%%%%%%%%%%%%%%%%%%%%%%%%%%%%%%%%%%%%%%%%%%%%%%%

\newpage

\section{Types `ramp' and `plat.ramp'}
\noindent These relations involve adding a gradual increase (or decrease) of probabitily between two inflection points. 
They are an 'upgraded' version of `full' and `plat.full'. Two or four inflection points are needed. 
The main difference between `ramp'(Figure \ref{fig7} \& \ref{fig8}) and `plat.ramp' (Figure \ref{fig9}, \ref{fig10} \& \ref{fig11}) 
types are the number inflection points.\\*
			
\begin{description}
\item[ramp.sel]\verb#(x, infl1, infl2, pos=TRUE, lv=0, uv=1)#
\item[plat.ramp.sel]\verb#(x, infl1, infl2, infl3, infl4, pos=TRUE,# \\* \verb#lv=c(0,0), uv=c(1,1))#
\end{description}
where \verb#x# is a numeric vector for which probabilities are estimated, \verb#infl1# to \verb#infl4# are the inflection points,  \verb#pos# indicates if the trend at the beginning is positive  (\verb#TRUE#) or negative (\verb#FALSE#), \verb#lv# defines the 
lower probability values of the relation, and \verb#uv# defines the upper probability values of the relation. By default, 
all fonctions have \verb#pos=TRUE#, \verb#lv=c(0,0)#, and \verb#uv=c(1,1)#.\\*

Here are examples for `ramp' type:
\begin{Hchunk}
\begin{normalsize}
\begin{Hinput}
\ttfamily\noindent
\hlprompt{\usebox{\hlnormalsizeboxgreaterthan}{\ }}\hlfunctioncall{ramp.sel}\hlkeyword{(}\hlargument{x}\hlargument{=}\hlsymbol{data}\hlkeyword{,}{\ }\hlargument{infl1}\hlargument{=}\hlnumber{1000}\hlkeyword{,}{\ }\hlargument{infl2}\hlargument{=}\hlnumber{2000}\hlkeyword{,}{\ }\hlargument{pos}\hlargument{=}\hlnumber{TRUE}\hlkeyword{)}\mbox{}
\normalfont
\end{Hinput}


\begin{Hinput}
\ttfamily\noindent
\hlprompt{\usebox{\hlnormalsizeboxgreaterthan}{\ }}\hlfunctioncall{ramp.sel}\hlkeyword{(}\hlargument{x}\hlargument{=}\hlsymbol{data}\hlkeyword{,}{\ }\hlargument{infl1}\hlargument{=}\hlnumber{1000}\hlkeyword{,}{\ }\hlargument{infl2}\hlargument{=}\hlnumber{2000}\hlkeyword{,}{\ }\hlargument{pos}\hlargument{=}\hlnumber{FALSE}\hlkeyword{)}\mbox{}
\normalfont
\end{Hinput}


\end{normalsize}
\end{Hchunk}


\begin{figure}[h]
\vspace{-20pt}
\begin{center}
\includegraphics{selectivity-015}
\end{center}
\vspace{-30pt}
\caption{Type 'ramp' probabilities (left -> pos=TRUE |  right -> pos=FALSE).}
\vspace{-10pt}
\label{fig7}
\end{figure}

\vspace*{\fill}
\newpage


\begin{Hchunk}
\begin{normalsize}
\begin{Hinput}
\ttfamily\noindent
\hlprompt{\usebox{\hlnormalsizeboxgreaterthan}{\ }}\hlfunctioncall{ramp.sel}\hlkeyword{(}\hlargument{x}\hlargument{=}\hlsymbol{data}\hlkeyword{,}{\ }\hlargument{infl1}\hlargument{=}\hlnumber{1000}\hlkeyword{,}{\ }\hlargument{infl2}\hlargument{=}\hlnumber{2000}\hlkeyword{,}{\ }\hlargument{pos}\hlargument{=}\hlnumber{TRUE}\hlkeyword{,}{\ }\hlargument{lv}\hlargument{=}\hlnumber{0.2}\hlkeyword{,}\hlargument{uv}\hlargument{=}\hlnumber{0.8}\hlkeyword{)}\mbox{}
\normalfont
\end{Hinput}


\begin{Hinput}
\ttfamily\noindent
\hlprompt{\usebox{\hlnormalsizeboxgreaterthan}{\ }}\hlfunctioncall{ramp.sel}\hlkeyword{(}\hlargument{x}\hlargument{=}\hlsymbol{data}\hlkeyword{,}{\ }\hlargument{infl1}\hlargument{=}\hlnumber{1000}\hlkeyword{,}{\ }\hlargument{infl2}\hlargument{=}\hlnumber{2000}\hlkeyword{,}{\ }\hlargument{pos}\hlargument{=}\hlnumber{FALSE}\hlkeyword{,}{\ }\hlargument{lv}\hlargument{=}\hlnumber{0.2}\hlkeyword{,}\hlargument{uv}\hlargument{=}\hlnumber{0.8}\hlkeyword{)}\mbox{}
\normalfont
\end{Hinput}


\end{normalsize}
\end{Hchunk}


\begin{figure}[h]
\vspace{-20pt}
\begin{center}
\includegraphics{selectivity-017}
\end{center}
\vspace{-30pt}
\caption{Type 'ramp' probabilities (left -> pos=TRUE |  right -> pos=FALSE). In this example, minimum and maximum probabilities are respectively lv=0.2 and uv=0.8.}
\vspace{-10pt}
\label{fig8}
\end{figure}

\newpage

Here are examples for `plat.ramp' type:
\begin{Hchunk}
\begin{normalsize}
\begin{Hinput}
\ttfamily\noindent
\hlprompt{\usebox{\hlnormalsizeboxgreaterthan}{\ }}\hlsymbol{data}\hlassignement{=}\hlnumber{0}\hlkeyword{:}\hlnumber{3000}\mbox{}
\normalfont
\end{Hinput}


\begin{Hinput}
\ttfamily\noindent
\hlprompt{\usebox{\hlnormalsizeboxgreaterthan}{\ }}\hlfunctioncall{plat.ramp.sel}\hlkeyword{(}\hlargument{x}\hlargument{=}\hlsymbol{data}\hlkeyword{,}{\ }\hlargument{infl1}\hlargument{=}\hlnumber{500}\hlkeyword{,}{\ }\hlargument{infl2}\hlargument{=}\hlnumber{1000}\hlkeyword{,}{\ }\hlargument{infl3}\hlargument{=}\hlnumber{2000}\hlkeyword{,}{\ }\hlargument{infl4}\hlargument{=}\hlnumber{2500}\hlkeyword{,}{\ }\hlargument{pos}\hlargument{=}\hlnumber{TRUE}\hlkeyword{)}\mbox{}
\normalfont
\end{Hinput}


\begin{Hinput}
\ttfamily\noindent
\hlprompt{\usebox{\hlnormalsizeboxgreaterthan}{\ }}\hlfunctioncall{plat.ramp.sel}\hlkeyword{(}\hlargument{x}\hlargument{=}\hlsymbol{data}\hlkeyword{,}{\ }\hlargument{infl1}\hlargument{=}\hlnumber{500}\hlkeyword{,}{\ }\hlargument{infl2}\hlargument{=}\hlnumber{1000}\hlkeyword{,}{\ }\hlargument{infl3}\hlargument{=}\hlnumber{2000}\hlkeyword{,}{\ }\hlargument{infl4}\hlargument{=}\hlnumber{2500}\hlkeyword{,}{\ }\hlargument{pos}\hlargument{=}\hlnumber{FALSE}\hlkeyword{)}\mbox{}
\normalfont
\end{Hinput}


\end{normalsize}
\end{Hchunk}

\begin{figure}[h]
\vspace{-20pt}
\begin{center}
\includegraphics{selectivity-019}
\end{center}
\vspace{-30pt}
\caption{Type 'plat.ramp' probabilities (left -> pos=TRUE |  right -> pos=FALSE).}
\vspace{-10pt}
\label{fig9}
\end{figure}


\begin{Hchunk}
\begin{normalsize}
\begin{Hinput}
\ttfamily\noindent
\hlprompt{\usebox{\hlnormalsizeboxgreaterthan}{\ }}\hlsymbol{data}\hlassignement{=}\hlnumber{0}\hlkeyword{:}\hlnumber{3000}\mbox{}
\normalfont
\end{Hinput}


\begin{Hinput}
\ttfamily\noindent
\hlprompt{\usebox{\hlnormalsizeboxgreaterthan}{\ }}\hlfunctioncall{plat.ramp.sel}\hlkeyword{(}\hlargument{x}\hlargument{=}\hlsymbol{data}\hlkeyword{,}{\ }\hlargument{infl1}\hlargument{=}\hlnumber{500}\hlkeyword{,}{\ }\hlargument{infl2}\hlargument{=}\hlnumber{1000}\hlkeyword{,}{\ }\hlargument{infl3}\hlargument{=}\hlnumber{2000}\hlkeyword{,}{\ }\hlargument{infl4}\hlargument{=}\hlnumber{2500}\hlkeyword{,}{\ }\hlargument{pos}\hlargument{=}\hlnumber{TRUE}\hlkeyword{,}{\ }\hlargument{lv}\hlargument{=}\hlnumber{0.2}\hlkeyword{,}\hlargument{uv}\hlargument{=}\hlnumber{0.8}\hlkeyword{)}\mbox{}
\normalfont
\end{Hinput}


\begin{Hinput}
\ttfamily\noindent
\hlprompt{\usebox{\hlnormalsizeboxgreaterthan}{\ }}\hlfunctioncall{plat.ramp.sel}\hlkeyword{(}\hlargument{x}\hlargument{=}\hlsymbol{data}\hlkeyword{,}{\ }\hlargument{infl1}\hlargument{=}\hlnumber{500}\hlkeyword{,}{\ }\hlargument{infl2}\hlargument{=}\hlnumber{1000}\hlkeyword{,}{\ }\hlargument{infl3}\hlargument{=}\hlnumber{2000}\hlkeyword{,}{\ }\hlargument{infl4}\hlargument{=}\hlnumber{2500}\hlkeyword{,}{\ }\hlargument{pos}\hlargument{=}\hlnumber{FALSE}\hlkeyword{,}{\ }\hlargument{lv}\hlargument{=}\hlnumber{0.2}\hlkeyword{,}\hlargument{uv}\hlargument{=}\hlnumber{0.8}\hlkeyword{)}\mbox{}
\normalfont
\end{Hinput}


\end{normalsize}
\end{Hchunk}

\begin{figure}[h]
\vspace{-20pt}
\begin{center}
\includegraphics{selectivity-021}
\end{center}
\vspace{-30pt}
\caption{Type 'plat.ramp' probabilities (left -> pos=TRUE |  right -> pos=FALSE). In this example, minimum and maximum probabilities are respectively lv=0.2 and uv=0.8.}
\vspace{-20pt}
\label{fig10}
\end{figure}

\begin{Hchunk}
\begin{normalsize}
\begin{Hinput}
\ttfamily\noindent
\hlprompt{\usebox{\hlnormalsizeboxgreaterthan}{\ }}\hlsymbol{data}\hlassignement{=}\hlnumber{0}\hlkeyword{:}\hlnumber{3000}\mbox{}
\normalfont
\end{Hinput}


\begin{Hinput}
\ttfamily\noindent
\hlprompt{\usebox{\hlnormalsizeboxgreaterthan}{\ }}\hlfunctioncall{plat.ramp.sel}\hlkeyword{(}\hlargument{x}\hlargument{=}\hlsymbol{data}\hlkeyword{,}{\ }\hlargument{infl1}\hlargument{=}\hlnumber{500}\hlkeyword{,}{\ }\hlargument{infl2}\hlargument{=}\hlnumber{1000}\hlkeyword{,}{\ }\hlargument{infl3}\hlargument{=}\hlnumber{2000}\hlkeyword{,}{\ }\hlargument{infl4}\hlargument{=}\hlnumber{2500}\hlkeyword{,}{\ }\hlargument{pos}\hlargument{=}\hlnumber{TRUE}\hlkeyword{,}{\ }\hlargument{lv}\hlargument{=}\hlfunctioncall{c}\hlkeyword{(}\hlnumber{0.2}\hlkeyword{,}\hlnumber{0.4}\hlkeyword{)}\hlkeyword{,}\hlargument{uv}\hlargument{=}\hlnumber{0.8}\hlkeyword{)}\mbox{}
\normalfont
\end{Hinput}


\begin{Hinput}
\ttfamily\noindent
\hlprompt{\usebox{\hlnormalsizeboxgreaterthan}{\ }}\hlfunctioncall{plat.ramp.sel}\hlkeyword{(}\hlargument{x}\hlargument{=}\hlsymbol{data}\hlkeyword{,}{\ }\hlargument{infl1}\hlargument{=}\hlnumber{500}\hlkeyword{,}{\ }\hlargument{infl2}\hlargument{=}\hlnumber{1000}\hlkeyword{,}{\ }\hlargument{infl3}\hlargument{=}\hlnumber{2000}\hlkeyword{,}{\ }\hlargument{infl4}\hlargument{=}\hlnumber{2500}\hlkeyword{,}{\ }\hlargument{pos}\hlargument{=}\hlnumber{FALSE}\hlkeyword{,}{\ }\hlargument{lv}\hlargument{=}\hlnumber{0.2}\hlkeyword{,}\hlargument{uv}\hlargument{=}\hlfunctioncall{c}\hlkeyword{(}\hlnumber{0.8}\hlkeyword{,}\hlnumber{0.6}\hlkeyword{)}\hlkeyword{)}\mbox{}
\normalfont
\end{Hinput}


\end{normalsize}
\end{Hchunk}

\begin{figure}[h]
\vspace{-20pt}
\begin{center}
\includegraphics{selectivity-023}
\end{center}
\vspace{-30pt}
\caption{Type 'plat.ramp' probabilities (left -> pos=TRUE, lv=c(0.2,0.4) , uv=0.8 | right -> pos=FALSE, lv=0.2, uv=c(0.8,0.6)).}
\vspace{-10pt}
\label{fig11}
\end{figure}

%%%%%%%%%%%%%%%%%%%%%%%%%%%%%%%%%%%%%%%%%%%%%%%%%%%%%%%%%%%%%%%%%%%%%%%%%%%%%%%%%%%%%%%%%%%%%%%%%%%
\newpage

\section{Types `logit' and `plat.logit'}
\noindent These relations use logistic curves. Inflection points are defined as points where the intantaneous splope 
is a proportion (\verb#prop#) of the intantenuous slope at $x_{50}$. These types make use of the function \verb#find.beta()# 
of \verb#package::bmisc#. Default value of \verb#prop# is \verb#0.1#. The end result is a logistic curve with 
$x_{50}$ being the midpoint between the inflection points. Two or four inflection points are needed. 
The main difference between `logit'(Figure \ref{fig12} \& \ref{fig13}) and `plat.logit' (Figure \ref{fig14}, \ref{fig15} \& \ref{fig16}) 
types are the number inflection points.\\*

\begin{description}
\item[logit.sel]\verb#(x, infl1, infl2, pos=TRUE, lv=0, uv=1, ...)#
\item[plat.logit.sel]\verb#(x, infl1, infl2, infl3, infl4, pos=TRUE,# \\* \verb#lv=c(0,0), uv=c(1,1), ...)#
\end{description}
where \verb#x# is a numeric vector for which probabilities are estimated, \verb#infl1# to \verb#infl4# are the inflection points,  \verb#pos# indicates if the trend at the beginning is positive  (\verb#TRUE#) or negative (\verb#FALSE#), \verb#lv# defines the 
lower probability values of the relation, and \verb#uv# defines the upper probability values of the relation. By default, 
all fonctions have \verb#pos=TRUE#, \verb#lv=c(0,0)#, and \verb#uv=c(1,1)#. Additionnal options of \verb#find.beta()# 
can be added. Default values are \verb#prob=NULL#, \verb#prop=0.1#,\verb#beta=0.2#, and \verb#fast=TRUE#.  \\*

Here are examples for `logit' type:
\begin{Hchunk}
\begin{normalsize}
\begin{Hinput}
\ttfamily\noindent
\hlprompt{\usebox{\hlnormalsizeboxgreaterthan}{\ }}\hlfunctioncall{logit.sel}\hlkeyword{(}\hlargument{x}\hlargument{=}\hlsymbol{data}\hlkeyword{,}{\ }\hlargument{infl1}\hlargument{=}\hlnumber{1000}\hlkeyword{,}{\ }\hlargument{infl2}\hlargument{=}\hlnumber{2000}\hlkeyword{,}{\ }\hlargument{pos}\hlargument{=}\hlnumber{TRUE}\hlkeyword{)}\mbox{}
\normalfont
\end{Hinput}


\begin{Hinput}
\ttfamily\noindent
\hlprompt{\usebox{\hlnormalsizeboxgreaterthan}{\ }}\hlfunctioncall{logit.sel}\hlkeyword{(}\hlargument{x}\hlargument{=}\hlsymbol{data}\hlkeyword{,}{\ }\hlargument{infl1}\hlargument{=}\hlnumber{1000}\hlkeyword{,}{\ }\hlargument{infl2}\hlargument{=}\hlnumber{2000}\hlkeyword{,}{\ }\hlargument{pos}\hlargument{=}\hlnumber{FALSE}\hlkeyword{)}\mbox{}
\normalfont
\end{Hinput}


\end{normalsize}
\end{Hchunk}

\begin{figure}[h]
\vspace{-20pt}
\begin{center}
\includegraphics{selectivity-025}
\end{center}
\vspace{-30pt}
\caption{Type 'logit' probabilities (left -> pos=TRUE |  right -> pos=FALSE).}
\vspace{-10pt}
\label{fig12}
\end{figure}

\vspace*{\fill}
\newpage


\begin{Hchunk}
\begin{normalsize}
\begin{Hinput}
\ttfamily\noindent
\hlprompt{\usebox{\hlnormalsizeboxgreaterthan}{\ }}\hlfunctioncall{logit.sel}\hlkeyword{(}\hlargument{x}\hlargument{=}\hlsymbol{data}\hlkeyword{,}{\ }\hlargument{infl1}\hlargument{=}\hlnumber{1000}\hlkeyword{,}{\ }\hlargument{infl2}\hlargument{=}\hlnumber{2000}\hlkeyword{,}{\ }\hlargument{pos}\hlargument{=}\hlnumber{TRUE}\hlkeyword{,}{\ }\hlargument{lv}\hlargument{=}\hlnumber{0.2}\hlkeyword{,}\hlargument{uv}\hlargument{=}\hlnumber{0.8}\hlkeyword{)}\mbox{}
\normalfont
\end{Hinput}


\begin{Hinput}
\ttfamily\noindent
\hlprompt{\usebox{\hlnormalsizeboxgreaterthan}{\ }}\hlfunctioncall{logit.sel}\hlkeyword{(}\hlargument{x}\hlargument{=}\hlsymbol{data}\hlkeyword{,}{\ }\hlargument{infl1}\hlargument{=}\hlnumber{1000}\hlkeyword{,}{\ }\hlargument{infl2}\hlargument{=}\hlnumber{2000}\hlkeyword{,}{\ }\hlargument{pos}\hlargument{=}\hlnumber{FALSE}\hlkeyword{,}{\ }\hlargument{lv}\hlargument{=}\hlnumber{0.2}\hlkeyword{,}\hlargument{uv}\hlargument{=}\hlnumber{0.8}\hlkeyword{)}\mbox{}
\normalfont
\end{Hinput}


\end{normalsize}
\end{Hchunk}


\begin{figure}[h]
\vspace{-20pt}
\begin{center}
\includegraphics{selectivity-027}
\end{center}
\vspace{-30pt}
\caption{Type 'logit' probabilities (left -> pos=TRUE |  right -> pos=FALSE). In this example, minimum and maximum probabilities are respectively lv=0.2 and uv=0.8.}
\vspace{-10pt}
\label{fig13}
\end{figure}

\newpage

Here are examples for `plat.logit' type:
\begin{Hchunk}
\begin{normalsize}
\begin{Hinput}
\ttfamily\noindent
\hlprompt{\usebox{\hlnormalsizeboxgreaterthan}{\ }}\hlsymbol{data}\hlassignement{=}\hlnumber{0}\hlkeyword{:}\hlnumber{3000}\mbox{}
\normalfont
\end{Hinput}


\begin{Hinput}
\ttfamily\noindent
\hlprompt{\usebox{\hlnormalsizeboxgreaterthan}{\ }}\hlfunctioncall{plat.logit.sel}\hlkeyword{(}\hlargument{x}\hlargument{=}\hlsymbol{data}\hlkeyword{,}{\ }\hlargument{infl1}\hlargument{=}\hlnumber{500}\hlkeyword{,}{\ }\hlargument{infl2}\hlargument{=}\hlnumber{1000}\hlkeyword{,}{\ }\hlargument{infl3}\hlargument{=}\hlnumber{2000}\hlkeyword{,}{\ }\hlargument{infl4}\hlargument{=}\hlnumber{2500}\hlkeyword{,}{\ }\hlargument{pos}\hlargument{=}\hlnumber{TRUE}\hlkeyword{)}\mbox{}
\normalfont
\end{Hinput}


\begin{Hinput}
\ttfamily\noindent
\hlprompt{\usebox{\hlnormalsizeboxgreaterthan}{\ }}\hlfunctioncall{plat.logit.sel}\hlkeyword{(}\hlargument{x}\hlargument{=}\hlsymbol{data}\hlkeyword{,}{\ }\hlargument{infl1}\hlargument{=}\hlnumber{500}\hlkeyword{,}{\ }\hlargument{infl2}\hlargument{=}\hlnumber{1000}\hlkeyword{,}{\ }\hlargument{infl3}\hlargument{=}\hlnumber{2000}\hlkeyword{,}{\ }\hlargument{infl4}\hlargument{=}\hlnumber{2500}\hlkeyword{,}{\ }\hlargument{pos}\hlargument{=}\hlnumber{FALSE}\hlkeyword{)}\mbox{}
\normalfont
\end{Hinput}


\end{normalsize}
\end{Hchunk}

\begin{figure}[h]
\vspace{-20pt}
\begin{center}
\includegraphics{selectivity-029}
\end{center}
\vspace{-30pt}
\caption{Type 'plat.logit' probabilities (left -> pos=TRUE |  right -> pos=FALSE).}
\vspace{-10pt}
\label{fig14}
\end{figure}


\begin{Hchunk}
\begin{normalsize}
\begin{Hinput}
\ttfamily\noindent
\hlprompt{\usebox{\hlnormalsizeboxgreaterthan}{\ }}\hlsymbol{data}\hlassignement{=}\hlnumber{0}\hlkeyword{:}\hlnumber{3000}\mbox{}
\normalfont
\end{Hinput}


\begin{Hinput}
\ttfamily\noindent
\hlprompt{\usebox{\hlnormalsizeboxgreaterthan}{\ }}\hlfunctioncall{plat.logit.sel}\hlkeyword{(}\hlargument{x}\hlargument{=}\hlsymbol{data}\hlkeyword{,}{\ }\hlargument{infl1}\hlargument{=}\hlnumber{500}\hlkeyword{,}{\ }\hlargument{infl2}\hlargument{=}\hlnumber{1000}\hlkeyword{,}{\ }\hlargument{infl3}\hlargument{=}\hlnumber{2000}\hlkeyword{,}{\ }\hlargument{infl4}\hlargument{=}\hlnumber{2500}\hlkeyword{,}{\ }\hlargument{pos}\hlargument{=}\hlnumber{TRUE}\hlkeyword{,}{\ }\hlargument{lv}\hlargument{=}\hlnumber{0.2}\hlkeyword{,}\hlargument{uv}\hlargument{=}\hlnumber{0.8}\hlkeyword{)}\mbox{}
\normalfont
\end{Hinput}


\begin{Hinput}
\ttfamily\noindent
\hlprompt{\usebox{\hlnormalsizeboxgreaterthan}{\ }}\hlfunctioncall{plat.logit.sel}\hlkeyword{(}\hlargument{x}\hlargument{=}\hlsymbol{data}\hlkeyword{,}{\ }\hlargument{infl1}\hlargument{=}\hlnumber{500}\hlkeyword{,}{\ }\hlargument{infl2}\hlargument{=}\hlnumber{1000}\hlkeyword{,}{\ }\hlargument{infl3}\hlargument{=}\hlnumber{2000}\hlkeyword{,}{\ }\hlargument{infl4}\hlargument{=}\hlnumber{2500}\hlkeyword{,}{\ }\hlargument{pos}\hlargument{=}\hlnumber{FALSE}\hlkeyword{,}{\ }\hlargument{lv}\hlargument{=}\hlnumber{0.2}\hlkeyword{,}\hlargument{uv}\hlargument{=}\hlnumber{0.8}\hlkeyword{)}\mbox{}
\normalfont
\end{Hinput}


\end{normalsize}
\end{Hchunk}

\begin{figure}[h]
\vspace{-20pt}
\begin{center}
\includegraphics{selectivity-031}
\end{center}
\vspace{-30pt}
\caption{Type 'plat.logit' probabilities (left -> pos=TRUE |  right -> pos=FALSE). In this example, minimum and maximum probabilities are respectively lv=0.2 and uv=0.8.}
\vspace{-20pt}
\label{fig15}
\end{figure}

\begin{Hchunk}
\begin{normalsize}
\begin{Hinput}
\ttfamily\noindent
\hlprompt{\usebox{\hlnormalsizeboxgreaterthan}{\ }}\hlsymbol{data}\hlassignement{=}\hlnumber{0}\hlkeyword{:}\hlnumber{3000}\mbox{}
\normalfont
\end{Hinput}


\begin{Hinput}
\ttfamily\noindent
\hlprompt{\usebox{\hlnormalsizeboxgreaterthan}{\ }}\hlfunctioncall{plat.logit.sel}\hlkeyword{(}\hlargument{x}\hlargument{=}\hlsymbol{data}\hlkeyword{,}{\ }\hlargument{infl1}\hlargument{=}\hlnumber{500}\hlkeyword{,}{\ }\hlargument{infl2}\hlargument{=}\hlnumber{1000}\hlkeyword{,}{\ }\hlargument{infl3}\hlargument{=}\hlnumber{2000}\hlkeyword{,}{\ }\hlargument{infl4}\hlargument{=}\hlnumber{2500}\hlkeyword{,}{\ }\hlargument{pos}\hlargument{=}\hlnumber{TRUE}\hlkeyword{,}{\ }\hlargument{lv}\hlargument{=}\hlfunctioncall{c}\hlkeyword{(}\hlnumber{0.2}\hlkeyword{,}\hlnumber{0.4}\hlkeyword{)}\hlkeyword{,}\hlargument{uv}\hlargument{=}\hlnumber{0.8}\hlkeyword{)}\mbox{}
\normalfont
\end{Hinput}


\begin{Hinput}
\ttfamily\noindent
\hlprompt{\usebox{\hlnormalsizeboxgreaterthan}{\ }}\hlfunctioncall{plat.logit.sel}\hlkeyword{(}\hlargument{x}\hlargument{=}\hlsymbol{data}\hlkeyword{,}{\ }\hlargument{infl1}\hlargument{=}\hlnumber{500}\hlkeyword{,}{\ }\hlargument{infl2}\hlargument{=}\hlnumber{1000}\hlkeyword{,}{\ }\hlargument{infl3}\hlargument{=}\hlnumber{2000}\hlkeyword{,}{\ }\hlargument{infl4}\hlargument{=}\hlnumber{2500}\hlkeyword{,}{\ }\hlargument{pos}\hlargument{=}\hlnumber{FALSE}\hlkeyword{,}{\ }\hlargument{lv}\hlargument{=}\hlnumber{0.2}\hlkeyword{,}\hlargument{uv}\hlargument{=}\hlfunctioncall{c}\hlkeyword{(}\hlnumber{0.8}\hlkeyword{,}\hlnumber{0.6}\hlkeyword{)}\hlkeyword{)}\mbox{}
\normalfont
\end{Hinput}


\end{normalsize}
\end{Hchunk}

\begin{figure}[h]
\vspace{-20pt}
\begin{center}
\begin{Houtput}
\ttfamily\noindent
{\ }{\ }{\ }{\ }{\ }beta{\ }{\ }{\ }alpha{\ }x50{\ }angle.x50{\ }min{\ }{\ }max{\ }angle.infl\hspace*{\fill}\\
\hlstd{}1{\ }0.01454{\ }-10.905{\ }750{\ }0.2082692{\ }500{\ }1000{\ }0.02086445\hspace*{\fill}\hlstd{}\mbox{}
\normalfont
\end{Houtput}
\includegraphics{selectivity-033}
\end{center}
\vspace{-30pt}
\caption{Type 'plat.logit' probabilities (left -> pos=TRUE, lv=c(0.2,0.4) , uv=0.8 | right -> pos=FALSE, lv=0.2, uv=c(0.8,0.6)).}
\vspace{-10pt}
\label{fig16}
\end{figure}







        
        
\end{document}
