\documentclass[a4paper]{book}
\usepackage[times,hyper]{Rd}
\usepackage{makeidx}
\usepackage[utf8,latin1]{inputenc}
\makeindex{}
\begin{document}
\chapter*{}
\begin{center}
{\textbf{\huge Package `YPR'}}
\par\bigskip{\large \today}
\end{center}
\begin{description}
\raggedright{}
\item[Type]\AsIs{Package}
\item[Title]\AsIs{What the package does (short line)}
\item[Version]\AsIs{0.1-0}
\item[Date]\AsIs{2011-10-03}
\item[Author]\AsIs{Benoit Bruneau}
\item[Maintainer]\AsIs{Benoit Bruneau }\email{benoit.bruneau1@gmail.com}\AsIs{}
\item[Depends]\AsIs{bmisc}
\item[Description]\AsIs{This is a length-based Yield Per Recruit (YPR) model. It will soon include a weight-based YPR component.}
\item[License]\AsIs{LGPL >= 3.0}
\item[LazyLoad]\AsIs{yes}
\end{description}
\Rdcontents{\R{} topics documented:}
\newpage
\inputencoding{utf8}
\HeaderA{plot.sel.ypr}{Plots of selectivity curves used in an \code{ypr} object}{plot.sel.ypr}
\keyword{\textbackslash{}textasciitilde{}kwd1}{plot.sel.ypr}
\keyword{\textbackslash{}textasciitilde{}kwd2}{plot.sel.ypr}
%
\begin{Usage}
\begin{verbatim}
plot.sel.ypr(object)
\end{verbatim}
\end{Usage}
%
\begin{Arguments}
\begin{ldescription}
\item[\code{object}] \code{ypr} object.

\end{ldescription}
\end{Arguments}
%
\begin{Author}\relax
Benoit Bruneau
\end{Author}
%
\begin{Examples}
\begin{ExampleCode}
ypr.mod=ypr(vonB=c(1564.5508, 0.1206),  
		LW=c(0.000000009434 , 3.042974620048),
		l.start=241.86,
		last.age=14,
		age.step=1,
		Fsel.type=list(type="ramp",infl1=650,infl2=900, lv=c(0,0.2)),
		F.max=2, 
		F.incr.YPR=0.0001,
		Msel.type=list(type="ramp", infl1=400, infl2=600, lv=0.5, pos=F), 
		M=0.2, 
		Mat=list(type="ramp",infl1=700, infl2=900),
		f.MSP=0.4) 



plot.sel(ypr.mod)
\end{ExampleCode}
\end{Examples}
\newpage
\inputencoding{utf8}
\HeaderA{plot.ypr}{Standard Yield per Recruit plot.}{plot.ypr}
%
\begin{Description}\relax
Yield per Recruit and Spawning Stock Biomass per Recruit are plotted with standard reference points. 
\end{Description}
%
\begin{Usage}
\begin{verbatim}
## S4 method for signature 'ypr'
plot(object, main, ylab.ypr, ylab.ssb, xlab, 
                     col.ypr, col.ssb, ref, legend)
\end{verbatim}
\end{Usage}
%
\begin{Arguments}
\begin{ldescription}
\item[\code{object}] an object of class \code{"ypr"} resulting from a
call to \code{\LinkA{ypr.l}{ypr.l}}

\item[\code{main}] main title for the graph

\item[\code{ylab.ypr}] a label for the YPR \code{y} axis

\item[\code{ylab.ssb}] a label for the SSB/R \code{y} axis

\item[\code{xlab}] a label for the YPR \code{x} axis.

\item[\code{col.ypr}] the color of the the color of the YPR line.

\item[\code{col.ssb}] the color of the the color of the SSB/R line.

\item[\code{ref}] logical; if \code{TRUE}, standard reference points are added to the plot.

\item[\code{legend}] logical; if \code{TRUE}, a legend is added in the \code{'topright'} corner of the plot.

\end{ldescription}
\end{Arguments}
%
\begin{Details}\relax

\bold{REFERENCE POINTS:}\\{}
Reference points used for result output are defined as follow:
\begin{itemize}

\item \bold{F.zero:}  F level when there is no fishing (F=0).
\item \bold{F.01:}    F level where the slope of yield curve is 10\% of the slope at \code{F.zero}. 
\item \bold{F.xx:}    F level where the MSP is at the level defined by \code{f.MSP} option. Default is 40\% (0.4).
\item \bold{F.max:}   F level where yield is maximum.

\end{itemize}

\end{Details}
%
\begin{Author}\relax
Benoit Bruneau
\end{Author}
%
\begin{SeeAlso}\relax
\code{\LinkA{ypr.l}{ypr.l}}
\end{SeeAlso}
%
\begin{Examples}
\begin{ExampleCode}

ypr.mod=ypr(vonB=c(1564.5508, 0.1206),  
		LW=c(0.000000009434 , 3.042974620048),
		l.start=241.86,
		last.age=14,
		age.step=1,
		Fsel.type=list(type="ramp",infl1=650,infl2=900, lv=c(0,0.2)),
		F.max=2, 
		F.incr.YPR=0.0001,
		Msel.type=list(type="ramp", infl1=400, infl2=600, lv=0.5, pos=F), 
		M=0.2, 
		Mat=list(type="ramp",infl1=700, infl2=900),
		f.MSP=0.4) 



plot(ypr.mod)

\end{ExampleCode}
\end{Examples}
\newpage
\inputencoding{utf8}
\HeaderA{rivard}{Rivard Weights Calculation}{rivard}
%
\begin{Description}\relax
This function applies Rivard equations to mid-year weight at age data to adjust values to Jan-1 basis. 
\end{Description}
%
\begin{Usage}
\begin{verbatim}
rivard(pds, pred=FALSE, K=2, plus.gr=FALSE)
\end{verbatim}
\end{Usage}
%
\begin{Arguments}
\begin{ldescription}
\item[\code{data}] 


\end{ldescription}
\end{Arguments}
%
\begin{Details}\relax
More to come.  Will be adding interpolation for spawning season.
\end{Details}
%
\begin{Author}\relax
Benoit Bruneau
FranC'ois GrC)goire
\end{Author}
%
\begin{Examples}
\begin{ExampleCode}

x=rnorm(30,800,10)
rivard(data.frame("2000"=x,"2001"=x*1.2, "2002"=x*0.8,"2003"=x*0.5))

\end{ExampleCode}
\end{Examples}
\newpage
\inputencoding{utf8}
\HeaderA{selectivity}{Selectivity functions}{selectivity}
\aliasA{const.sel}{selectivity}{const.sel}
\aliasA{full.sel}{selectivity}{full.sel}
\aliasA{logit.sel}{selectivity}{logit.sel}
\aliasA{mod.logit.sel}{selectivity}{mod.logit.sel}
\aliasA{plat.full.sel}{selectivity}{plat.full.sel}
\aliasA{plat.logit.sel}{selectivity}{plat.logit.sel}
\aliasA{plat.ramp.sel}{selectivity}{plat.ramp.sel}
\aliasA{ramp.sel}{selectivity}{ramp.sel}
%
\begin{Description}\relax
These selectivity functions are called by \code{\LinkA{ypr}{ypr}()}. They estimate probabilities [0,1] for a given functional shape and a given number of inflection points .
\end{Description}
%
\begin{Usage}
\begin{verbatim}
const.sel(x)
full.sel(x, infl1, pos=TRUE)
plat.full.sel(x, infl1, infl2, pos=TRUE)
ramp.sel(x, infl1, infl2, pos=TRUE)
plat.ramp.sel(x, infl1, infl2, infl3, infl4, pos=TRUE)
logit.sel(x, infl1, infl2, pos=TRUE, ...)
plat.logit.sel(x, infl1, infl2, pos=TRUE, ...)
mod.logit.sel(x, alpha, beta)
\end{verbatim}
\end{Usage}
%
\begin{Arguments}
\begin{ldescription}
\item[\code{x}] a numeric vector for which probabilities are to be estimated.

\item[\code{infl1 to infl4}] numeric value of the inflection point(s).

\item[\code{pos}] logical. Indicates if the trend at the beginning is positive  (\code{TRUE}) or negative (\code{FALSE})

\item[\code{...}] arguments to be passed to \code{\LinkA{find.beta}{find.beta}()}

\end{ldescription}
\end{Arguments}
%
\begin{Details}\relax
More to come.
\end{Details}
%
\begin{Author}\relax
Benoit Bruneau
\end{Author}
%
\begin{SeeAlso}\relax
\code{\LinkA{ypr}{ypr}}, \code{\LinkA{find.beta}{find.beta}}
\end{SeeAlso}
%
\begin{Examples}
\begin{ExampleCode}

library(bmisc)

x=0:1000

plot(full.sel(infl1=600,x=0:1000,lv=0,uv=1), ylim=c(0,1), type='l', lwd=3)

plot(plat.full.sel(infl1=300, infl2=800,x=x,lv=c(0.4), uv=0.9, neg=F) ~ x, ylim=c(0,1), type='l', lwd=3)

plot(ramp.sel(200,600,x=0:1000, lv=0.1,uv=1, neg=T), ylim=c(0,1), type='l', lwd=3)

plot(plat.ramp.sel(infl1=100,infl2=300,infl3=600,infl4=800,x=0:1000, lv=c(0.3,0.1), uv=c(0.7,0.5), neg=F), ylim=c(0,1), type='l', lwd=3)

plot(logit.sel(infl1=300,infl2=500,x=0:1000, lv=0.5, uv=1, neg=T), ylim=c(0,1), type='l', lwd=3)

plot(plat.logit.sel(infl1=200,infl2=400,infl3=600,infl4=800,x=0:1000, lv=c(0.2,0.8)), ylim=c(0,1), type='l', lwd=3)
\end{ExampleCode}
\end{Examples}
\newpage
\inputencoding{utf8}
\HeaderA{summary.ypr}{Summarizing the results of YPR models.}{summary.ypr}
%
\begin{Description}\relax
Summary for an object of class \code{"ypr"}.
\end{Description}
%
\begin{Usage}
\begin{verbatim}
## S4 method for signature 'ypr'
summary(object)
\end{verbatim}
\end{Usage}
%
\begin{Arguments}
\begin{ldescription}
\item[\code{object}] an object of class \code{"ypr"} resulting from a
call to \code{\LinkA{ypr.l}{ypr.l}}.

\end{ldescription}
\end{Arguments}
%
\begin{Author}\relax
Benoit Bruneau
\end{Author}
%
\begin{Examples}
\begin{ExampleCode}
ypr.mod = ypr.l(fsel.type=list("ramp",650,900), vonB=c(1564.512,0.1205765), 
                l.start=72.53,last.age=25,LW=c(exp( -18.47894),3.043), F.max=2,
                F.incr.YPR=0.01, M=0.2, mat=list('full',900), f.MSP=0.4, riv.calc=F)  

summary(ypr.mod)
\end{ExampleCode}
\end{Examples}
\newpage
\inputencoding{utf8}
\HeaderA{YPR}{Yield Per Recruit (YPR) model.}{YPR}
%
\begin{Description}\relax
This is a length-based Yield Per Recruit (YPR) model. It will soon include a weight-based YPR component.
\end{Description}
%
\begin{Details}\relax

\Tabular{ll}{
Package: & YPR\\{}
Type: & Package\\{}
Version: & 0.1-1\\{}
Date: & 2011-10-03\\{}
License: & LGPL >= 3.0\\{}
LazyLoad: & yes\\{}
}

\end{Details}
%
\begin{Author}\relax
Benoit Bruneau
FranC'ois GrC)goire
Diane Archambault

Maintainer: Benoit Bruneau <benoit.bruneau1@gmail.com>
\end{Author}
\newpage
\inputencoding{utf8}
\HeaderA{ypr}{Length Based Yield Per Recruit }{ypr}
%
\begin{Description}\relax
Length based Yield Per Recruit model is define by fishery selectivity and life history parameters related to length. 


\end{Description}
%
\begin{Usage}
\begin{verbatim}
 ypr(LW, vonB, l.start, last.age, age.step=1, Fsel.type, 
       F.max=2,F.incr.YPR=0.0001, Mat,  M=0.2, f.MSP=0.4)
       
\end{verbatim}
\end{Usage}
%
\begin{Arguments}
\begin{ldescription}
\item[\code{LW}] one of:
\begin{itemize}

\item a vector containing \code{c(\eqn{\alpha}{},\eqn{\beta}{})} from length-weight curve. See `Details'. 
\item an object of class \code{"\LinkA{nls}{nls}"} in which \eqn{\alpha}{} and \eqn{\beta}{} were estimated. See `Details'.        

\end{itemize}


\item[\code{vonB}] one of:
\begin{itemize}

\item a vector containing \code{c(Linf,K)} from von Bertalanffy grotwh curve.
\item an object of class \code{"\LinkA{glm}{glm}"} in which eqnLinf and eqnK were estimated.      

\end{itemize}


\item[\code{l.start}] length at the starting age

\item[\code{last.age}] last age to be considered in the model

\item[\code{age.step}] steps used to generate ages. Default is \code{1}.


\item[\code{Fsel.type}] fishing selectivity can be defined as one of:
\begin{itemize}

\item a list containing the type of fishery selectivity and the values needed for the function related to the type.
\item an object of class \code{"\LinkA{glm}{glm}"} in which \eqn{\alpha}{} and \eqn{\beta}{} were estimated by a logistic regression.        

\end{itemize}


\item[\code{fish.lim}] the minimum legal catch length.

\item[\code{prop.surv}] a function that defines the proportion of fish (< \code{fish.lim}) that will survive after being released back into the water (discarded by-catch).

\item[\code{F.max}] maximum value of instantaneous rate of fishing mortality (\code{F}). Default is 2.

\item[\code{F.incr.YPR}] increment for generating the \code{F} values to be used for YPR calculation. Default is 0.0001.

\item[\code{Mat}] Maturity can be defined by one of:
\begin{itemize}

\item a list containing the type of maturity at length definition and the values needed for the function related to the type.
\item an object of class \code{"\LinkA{glm}{glm}"} in which \eqn{\alpha}{} and \eqn{\beta}{} were estimated by a logistic regression.       

\end{itemize}


\item[\code{Msel.type}] natural mortality selectivity can be defined as one of:
\begin{itemize}

\item a list containing the type of natural mortality selectivity and the values needed for the function related to the type. 
\item an object of class \code{"\LinkA{glm}{glm}"} in which \eqn{\alpha}{} and \eqn{\beta}{} were estimated by a logistic regression.       

\end{itemize}


\item[\code{M}] instantaneous rate of natural mortality (\code{M}). Default is 0.2.

\item[\code{f.MSP}] reference point defined as the fraction of maximum spawning potential. Default is 0.4.







\end{ldescription}
\end{Arguments}
%
\begin{Details}\relax
\bold{LENGTH-WEIGHT RELATIONSHIP:}\\{}
Length-Weight relationship can be provided either by indicating \code{c(\eqn{\alpha}{},\eqn{\beta}{})} values in a vector or by 
directly using an object of class \code{"\LinkA{nls}{nls}"} or \code{"\LinkA{lm}{lm}"}. If \eqn{\alpha}{} and \eqn{\beta}{} are estimated 
by \code{\LinkA{lm}{lm}}, \code{\LinkA{log}{log}(x, base=exp(1))} transformation should be applied to the data prior to fitting the linear model. 
If \eqn{\alpha}{} and \eqn{\beta}{} are estimated by \code{\LinkA{nls}{nls}}, variables should be named \bold{alpha} and \bold{beta} 
using the following equation:

\Tabular{c}{
\eqn{W=\alpha L^\beta}{}
}
where \eqn{W}{} is weight, \eqn{L}{} is length, \eqn{\alpha}{} is the elevation of the curve, and \eqn{\beta}{} 
is the steepness of the curve. Both \eqn{\alpha}{} and \eqn{\beta}{} are estimated coefficients.\\{}\\{}


\bold{VON BARTANLANFFY GROWTH EQUATION:}\\{}
Von Bartalanffy growth equation parameters can be provided either by indicating \code{c(Linf,K)} values in a vector or by 
directly using an object of class \code{"\LinkA{nls}{nls}"}. If an object resulting from \code{\LinkA{nls}{nls}} is used, variables 
should be named \bold{Linf} and \bold{K}. As for \bold{\eqn{t0}{}}, any name may be used since only \eqn{L\infty }{} and \eqn{K}{} are 
used in this length-based YPR model because age is considered as relative. The equation used in the \code{\LinkA{nls}{nls}} for estimating \eqn{L\infty }{} 
and \eqn{K}{} should be the following one:


\Tabular{c}{
\eqn{ L_t=L_{\infty }\bigg{(} 1-e^{-K (t-t_0)}\bigg{)} }{}
}
where \eqn{Lt}{} is length-at-age \eqn{t}{}, \eqn{L\infty}{} is the asymptotic average maximum length, 
\eqn{K}{} is a growth rate coefficient determinant of how quick the maximum is attained, and \eqn{t0}{} 
is the hypothetical age at length zero.

As stated above, since this length-based YPR model uses relative age, \eqn{t-t0}{} becomes a relative age (\eqn{a}{}). 
The Von Bartalanffy growth equation used in this length-based YPR model is defined as:

\Tabular{c}{
\eqn{ L_a=L_{\infty }\bigg{(} 1-e^{-Ka}\bigg{)}+ L_s e^{-Ka} }{}
}
where \eqn{L_a}{} is length at a relative age \eqn{a}{} and \eqn{L_s}{} is length at relative age zero.\\{}\\{}


\bold{SELECTIVITY CURVES:}\\{}
The \bold{fishery selectivity}, \bold{natural mortality selectivity}, and \bold{maturity at length} components of the 
model can be defined as one of \code{c("full", "plat.full", "ramp", "plat.ramp", "logit", "plat.logit")} equations. The 
proper way to specify which function to use is by the construct of a \code{list} where the first element is the 
name of one of the six types of function. See example, read \code{\LinkA{selectivity}{selectivity}}, or read \code{vignette("selectivity")} 
for more details.  

Alternatively, an object of class \code{"glm"} can directly be used for the \bold{fishery selectivity} and \bold{maturity at length} components. 
The Generalized Linear Model should have  the option \code{family} set to either \code{binomial} or \code{quasibinomial} keeping link function to the 
default (\eqn{i.e.}{} \code{"logit"}).
Estimated coefficients are use as follow: 

\Tabular{c}{
\eqn{y=\frac{1}{1+e^{-(\alpha+\beta x)}}}{}\\{}\\{} 
}

\bold{REFERENCE POINTS:}\\{}
Reference points used for result output are defined as follow:
\begin{itemize}

\item \bold{F.zero:}  F level when there is no fishing (F=0).
\item \bold{F.01:}    F level where the slope of yield curve is 10\% of the slope at \code{F.zero}. 
\item \bold{F.xx:}    F level where the MSP is at the level defined by \code{f.MSP} option. Default is 40\% (0.4).
\item \bold{F.max:}   F level where yield is maximum.

\end{itemize}


Use \code{vignette("YPR")} for a better presentation of the equations.      

\end{Details}
%
\begin{Value}
  
\code{ypr} returns an object of \code{class(S4)} \code{"ypr"}. The functions \code{summary}, \code{plot.sel.ypr}, and \code{plot} are used to respectively
obtain a summary, plots for selectivity curves used, and a standard YPR plot of the results.  

An object of class \code{"ypr"} has the the following slots:


\begin{ldescription}
\item[\code{parms}] the list of parameters used in the model.
\item[\code{base}] a \code{data.frame} containing the starting values:
\begin{itemize}

\item relative age classes
\item length at age
\item weight at age

\end{itemize}


\item[\code{refs}] a \code{data.frame} containing values predicted by the model for the four reference points. See details. 
\item[\code{YPR}] a \code{data.frame} containing the results for all partial Fs.


\end{ldescription}
Note that to have access to each slot of an \code{"ypr"} object, one must use \code{"@"} instead of \code{"\$"}.
\end{Value}
%
\begin{Author}\relax
Benoit Bruneau
FranC'ois GrC)goire
Diane Archambault
\end{Author}
%
\begin{SeeAlso}\relax
\code{\LinkA{plot.logit}{plot.logit}}, \code{\LinkA{plot.ypr}{plot.ypr}} and \code{\LinkA{plot.parms.ypr}{plot.parms.ypr}}  
\end{SeeAlso}
%
\begin{Examples}
\begin{ExampleCode}

surv=function(x){
    p=x
    p[x<910]=0.8
    p[x>=910]=0.2
    p
}

ypr(vonB=c(1564.5508, 0.1206),  
        LW=c(0.000000009434 , 3.042974620048),
        l.start=241.86,
        last.age=14,
        age.step=1,
        prop.surv=surv,
        fish.lim=1000,
        Fsel.type=list(type="ramp",infl1=650,infl2=900, lv=c(0,0.2)),
        F.max=2, 
        F.incr.YPR=0.0001,
        Msel.type=list(type="ramp", infl1=400, infl2=600, lv=0.5, pos=F), 
        M=0.2, 
        Mat=list(type="ramp",infl1=700, infl2=900),
        f.MSP=0.4) 

\end{ExampleCode}
\end{Examples}
\printindex{}
\end{document}
